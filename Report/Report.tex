\documentclass{article}
\usepackage[italian]{babel}
\usepackage[utf8]{inputenc}
\usepackage[margin=75pt]{geometry}
\usepackage{amsmath}

%IMPORT CODE
%\usepackage{minted}
\usepackage{listings}
\usepackage{color}

\definecolor{dkgreen}{rgb}{0,0.6,0}
\definecolor{gray}{rgb}{0.5,0.5,0.5}
\definecolor{mauve}{rgb}{0.58,0,0.82}

\lstset{frame=tb,
	language=Haskell,
	aboveskip=3mm,
	belowskip=3mm,
	showstringspaces=false,
	columns=flexible,
	basicstyle={\small\ttfamily},
	numbers=none,
	numberstyle=\tiny\color{gray},
	keywordstyle=\color{blue},
	commentstyle=\color{dkgreen},
	stringstyle=\color{mauve},
	breaklines=true,
	breakatwhitespace=true,
	tabsize=3
}

% in way to comment a block
\long\def\/*#1*/{}

\title{\textbf{Relazione sul Progetto d'Esame}}
\author{Francesco Rombaldoni}
\date{\small Università degli Studi di Urbino Carlo Bo\\
	Insegnamento di Programmazione Logica e Funzionale}

\begin{document}
	\maketitle
	
%	Prepared for a formal document
%	\newpage
	
%	\makeindex
%	\newpage

\section{Specifica del Problema}

Scrivere un programma Haskell e un programma Prolog che calcolino le proprietà di rotta formata da due punti satellitari in formato D.M.G.
\newline
\newline
\emph{[Confrontare lo svolgimento delle fasi successive con quanto riportato nelle prossime pagine.]}\\
\newpage
			
\section{Analisi del Problema}
\subsection{Dati di Ingresso del Problema}
\subsection{Dati di Uscita del Problema}
\subsection{Relazioni Intercorrenti tra i Dati del Problema}
\newpage

\section{Progettazione dell'Algoritmo}
\subsection{Scelte di Progetto}
\subsection{Passi dell'Algoritmo}
\newpage

\section{Implementazione dell'Algoritmo}
\raggedright
\underline{File sorgente \textbf{nome\_file.hs:}}
\lstset{language=Haskell}
\begin{lstlisting}
	-- Codice di haskell
	bla bla bla
\end{lstlisting}
\newpage
\raggedright
\underline{File sorgente \textbf{nome\_file.pl:}}
\lstset{language=Prolog}
\begin{lstlisting}
	/* Codice di prolog*/
\end{lstlisting}
\newpage

\section{Testing del Programma}
\subsection*{Test Haskell 1}
\subsection*{Test Haskell 2}
\subsection*{Test Haskell 3}
\subsection*{Test Haskell 4}
\subsection*{Test Haskell 5}
\subsection*{Test Haskell 6}
\subsection*{Test Haskell 7}
\subsection*{Test Haskell 8}
\subsection*{Test Haskell 9}
\subsection*{Test Haskell 10}
\newpage
\subsection*{Test Prolog 1}
\subsection*{Test Prolog 2}
\subsection*{Test Prolog 3}
\subsection*{Test Prolog 4}
\subsection*{Test Prolog 5}
\subsection*{Test Prolog 6}
\subsection*{Test Prolog 7}
\subsection*{Test Prolog 8}
\subsection*{Test Prolog 9}
\subsection*{Test Prolog 10}
\newpage

\section{Considerazioni Finali}
\raggedright
\underline{File sorgente \textbf{nome\_file.hs} esteso con l'azione iniziale:}
\lstset{language=Haskell}
\begin{lstlisting}
	-- Codice di haskell
\end{lstlisting}
% \newline
.
\newline
.
\newline
.
\newline
Testo di commento finale.
\newpage
\raggedright
\underline{File sorgente \textbf{nome\_file.pl} esteso con l'azione iniziale:}
\lstset{language=Prolog}
\begin{lstlisting}
	/*Codice di prolog*/
\end{lstlisting}
 %\newline
.
\newline
.
\newline
.
\newline
Testo di commento finale.
\end{document}